\documentclass[11pt]{article}
\usepackage[tmargin=3cm,lmargin=3cm,rmargin=3cm,bmargin=3cm]{geometry}
\usepackage{graphicx} % Required for inserting images
\usepackage[authordate]{biblatex-chicago}
\addbibresource{bib.bib}
\usepackage{listings}
\usepackage[htt]{hyphenat}

\lstnewenvironment{listing}{%
    \lstset{
      language={},
      basicstyle=\ttfamily,
      extendedchars=true,
        breaklines=true,
        breakatwhitespace=true,
        breakindent=0pt,
        % border:
        frame=single,
        framerule=0.25pt,
        framesep=5pt,
        xleftmargin=6pt,
        xrightmargin=6pt,
        caption={Description of formal language for conveying meaning of user utterances in the domain at hand.},captionpos=b,label={listing:L}
    }
}{}

\title{LLM-Based NLU for Explanatory Dialogue}
\author{alexander.berman@gu.se}

\begin{document}

\maketitle

\section{Introduction}
This report presents and discusses the integration of a large language model (LLM) based natural-language understanding (NLU) component into an explanatory dialogue system. The dialogue system is a chatbot that assists users in a game that revolves around assessing whether people are introverted or extraverted based on their music preferences. This chatbot is embedded in an interactive web page together with other elements of the game such as visualizations of music preferences, audio controls and navigation buttons; however, this report focuses only on NLU. (The game will be further elaborated in the course project report.)

\section{Motivation}
One of the reasons for opting for an LLM-based NLU component is to be able to interpret user utterances in a more accurate and robust way than with a simple keyword- or rule-based component. Furthermore, it was hypothesized that an LLM-based solution would require less effort in terms of development time than a more conventional machine-learning based approach (such as logistic regression or support vector machine for intent classification) due to the ability to use zero- or few-shot learning, which in principle can drastically reduce the amount of required training data. Finally, it was deemed that the fine-grained semantics involved in the modelling of explanatory dialogue phenomena would not be supported well by a more conventional approach based on intent classification and entity extraction. With an LLM, the problem can be approached as a sequence-to-sequence task, i.e. as a transformation from natural language to an expression in formal language whose syntax and semantics is fed to the LLM in the prompt.

\section{Problem formulation}
In conventional approaches, NLU for dialogue systems is typically divided into two ``tasks'': intent classification and entity extraction. For example, the user utterance ``I need a ticket to Paris'' can be understood as expressing the intent \texttt{BookTicket} and the entity \texttt{Paris} (possibly labelled with the role \texttt{Destination}). In contrast, here the task as framed as a matter of expressing the meaning of a given natural-language utterance in a formal language. For example, the meaning of the utterance above might be expressed as a sequence of dialogue moves such as [\texttt{Request(BookTicket)}, \texttt{Assert(Destination(Paris))}].

In general formal terms, given a user utterance $U$ and a description $L$ of a formal language, the NLU component should return an expression $E$ in $L$ that conveys the meaning of $U$ (where $U$, $L$ and $E$ are strings or token sequences). The value of $L$ in the context of the specific domain is provided in Listing \ref{listing:L} (for source code, see \texttt{project/dialog/music\_personality/nlu\_openai\_config.yml}).

\begin{listing}
The output can be one of the following:
- Ask(Question): the user asks a question
- ICM(understanding, negative): the user signals negative understanding
- ICM(acceptance, positive): the user acknowledges
- None: the input from the user does not match any of the above

Question can be one of:
- Why(): bare why question
- Why(Proposition): explicit why question concerning a proposition
- BooleanQuestion(Proposition): question concerning whether a proposition is true
- WhQuestion(FactorsConsidered): question concerning which factors the system considers

Proposition can be one of:
- Extraverted(): the person is considered extraverted
- Not(Extraverted()): the person is considered introverted
- HighValue(X): the person likes music with high X, where X is an audio feature
- Not(HighValue(X)): the person likes music with low X, where X is an audio feature
- Explains(Explanans, Explanandum): Explanans explains Explanandum (both being propositions)
- Supports(Antecedent, Consequent): Antecedent supports Consequent (both being propositions)

Audio features:
- energy_mean
- mode_0_percentage (low values indicate a preference for music with minor mode, while high values indicate a preference for major mode)
- loudness_mean
- speechiness_mean
- instrumentalness_mean
- valence_mean
- danceability_mean
\end{listing}


\section{Prompt design}
The prompt contains a description of the task and of the formal language as well as examples of inputs and outputs, and has been iterated in tandem with testing (see section \ref{sec:evaluation}). The most recent version of the prompt is shown in Listing \ref{listing:prompt}.

\lstnewenvironment{prompt}{%
    \lstset{
      language={},
      basicstyle=\ttfamily,
      extendedchars=true,
        breaklines=true,
        breakatwhitespace=true,
        breakindent=0pt,
        % border:
        frame=single,
        framerule=0.25pt,
        framesep=5pt,
        xleftmargin=6pt,
        xrightmargin=6pt,
        caption={Most recent version of the prompt. (L) is a slot for the description of the formal language (see Listing \ref{listing:L}).},captionpos=b,label={listing:prompt}
    }
}{}

\begin{prompt}
Your task is to interpret the meaning of a user utterance and output it in a particular semantic representation. Only output the semantic representation without any surrounding content.

SEMANTIC REPRESENTATION

(L)

EXAMPLES

why? -> Ask(Why())
why do you think this person is introverted -> Ask(Why(Not(Extraverted())))
why do you think this person likes danceable music? -> Ask(Why(HighValue(danceability_mean)))
why do you think this person likes music that is not danceable? -> Ask(Why(Not(HighValue(danceability_mean))))
why do you think this person likes loud music? -> Ask(Why(HighValue(loudness_mean)))
why do you think this person likes silent music? -> Ask(Why(Not(HighValue(loudness_mean))))
I don't understand -> ICM(understanding, negative)
so what? -> ICM(understanding, negative)
OK -> ICM(acceptance, positive)
how does liking music that is not danceable explain being introverted? -> Ask(Why(Explains(Not(HighValue(danceability_mean)), Not(Extraverted()))))
does the the fact that the person likes music that is not danceable support the assessment that the person is extraverted? -> Ask(BooleanQuestion(Supports(Not(HighValue(danceability_mean)), Extraverted())))
Which audio properties do you consider? -> Ask(WhQuestion(FactorsConsidered))
Does listening to Patti Smith affect your assessment? -> Ask(WhQuestion(FactorsConsidered))
do you think this person is extraverted? -> Ask(BooleanQuestion(Extraverted()))
next case please -> None
increase the volume -> None
\end{prompt}


\section{Implementation}
OpenAI's chat completion API is used as follows: Given the user utterance $U$, a prompt describing the task is provided as an initial system message, with $L$ embedded into the prompt. $U$ is then fed as a user message. As completion, an assistant message $M$ is obtained as output (see \texttt{project/dialog/music\_personality/nlu\_openai.py}). 

In order for the output $M$ to be used as input to the dialogue manager, it also needs to be parsed from a string (semantic expression) into a dialogue move. The dialogue system at hand is implemented in Python and uses custom types and classes to represent dialogue moves and their content. For example, instances of the class \texttt{Ask} denote ask-moves, and contain an instance of the class \texttt{Question}. To streamline the process of parsing semantic expressions, $L$ is formulated in way that literally reflects the dialogue system's internal representations. Hence, output from the LLM can be treated as executable Python code and can in principle be parsed by passing it to the built-in Python function \texttt{eval}. However, such a solution would not be secure since the output from LLM cannot be guaranteed to abide to the instructed form and content. Instead, a secure parser has been implemented. This parser uses Python's \texttt{ast} module, which enables Python code to be parsed into an abstract syntax tree. The parser iterates this tree recursively to construct a semantic structure. If the semantic expression being parsed is syntactically invalid or contains unsupported (and potentially unsafe) constructions, an exception is raised (see \texttt{project/dialog/semantic\_serialization.py}) which causes the interpretation to contain a \texttt{None} move; in such a situation, the system gives negative understanding feedback (``Sorry, I don't understand'').

\section{Evaluation}
\label{sec:evaluation}
At this point, the integration has only been evaluated unsystematically by manually testing the dialogue system in an end-to-end manner. Initially, it was noted that the LLM (GPT 3.5) accompanied the formal expressions with elaborations in natural language. This problem was solved by adding an instruction to the prompt to ``(o)nly output the semantic representation without any surrounding content.'' However, the model would still occassionally generate syntactically invalid output (mismatched parentheses). After switching to GPT 4, no occurrences of invalid syntax have been encountered.

As for accuracy, the impression so far from own manual testing is that GPT 4 performs very well. An example of a dialog collected during manual testing can be found in Listing \ref{listing:dialogexample}. One interesting observation concerns deixis: when the user utters ``why does that make it more likely that the person is extraverted?'' in response to the system asserting that the person likes silent music, GPT produces the following interpretation: \texttt{Ask(Why(Supports(Antecedent, Extraverted())))}. Here, the first argument to \texttt{Supports} can be taken to be underspecified (\texttt{Antecedent} is an argument name, rather than an instantiated proposition). Semantically, this seems to reflect the function of to the deictic demonstrative ``that'' in the user utterance. In other words, GPT captures the deictic function of ``that'' in a relevant way, without having seen any examples of this kind. (While the dialogue manager currently does not support underspecified semantic content of this kind, such an extension could be useful. Alternatively, one can consider letting the LLM resolve deixis by feeding dialogue history as input.)

\lstnewenvironment{dialogexample}{%
    \lstset{
      language={},
      basicstyle=\ttfamily,
      extendedchars=true,
        breaklines=true,
        breakatwhitespace=true,
        breakindent=0pt,
        % border:
        frame=single,
        framerule=0.25pt,
        framesep=5pt,
        xleftmargin=6pt,
        xrightmargin=6pt,
        caption={Example of dialog between user (U) and assistant (A), with NLU interpretations in square brackets.},captionpos=b,label={listing:dialogexample}
    }
}{}

\begin{dialogexample}
A: If I had to guess, I'd say that this person is extraverted.
U: which variables to you look at?
  [Ask(WhQuestion(FactorsConsidered))]
A: I consider music heard by the person in terms of the following audio features: energy, mode, loudness, speechiness, instrumentalness, valence and danceability.
U: why
  [Ask(Why())]
A: I don't know.
U: ok
  [ICM(acceptance, positive)]
(A: no response)
U: so you do think this person is extraverted?
  [Ask(BooleanQuestion(Extraverted()))]
A: Yes, but I'm very uncertain.
U: sorry, I meant: why  do think this person is extraverted?
  [Ask(Why(Extraverted()))]
A: The person likes music with low energy.
U: hmm, ok
  [ICM(acceptance, positive)]
(A: no response)
U: so low energy is associated with being introverted?
  [Ask(BooleanQuestion(Supports(Not(HighValue(energy_mean)), Not(Extraverted()))))]
A: No.
U: right, my bad, low energy is associated with extraversion?
  [Ask(BooleanQuestion(Supports(Not(HighValue(energy_mean)), Extraverted())))]
A: Yes.
U: gotcha
  [ICM(acceptance, positive)]
(A: no response)
\end{dialogexample}


\section{Conclusions}
The process of developing and evaluating an LLM-based NLU component for an explanatory dialogue system indicates that the direction taken makes it possible to leverage the rich linguistic and extralinguistic capabilities of modern pre-trained LLMs for conversational language understanding, without any of the disadvantages associated with LLM-based end-to-end or NLG applications (such as ``hallucinations'' or toxic content).

In future work it would be useful to evaluate performance more systematically, e.g. by collecting dialogue data, including user utterances and the LLM's interpretations, and annotating expected (correct) interpretations. This way, accuracy can be measured and be compared with other approaches.

\end{document}
