\lstnewenvironment{dialogexample}{%
    \lstset{
      language={},
      basicstyle=\ttfamily,
      extendedchars=true,
        breaklines=true,
        breakatwhitespace=true,
        breakindent=0pt,
        % border:
        frame=single,
        framerule=0.25pt,
        framesep=5pt,
        xleftmargin=6pt,
        xrightmargin=6pt,
        caption={Example of dialog between user (U) and assistant (A), with NLU interpretations in square brackets.},captionpos=b,label={listing:dialogexample}
    }
}{}

\begin{dialogexample}
A: If I had to guess, I'd say that this person is extraverted.
U: which variables to you look at?
  [Ask(WhQuestion(FactorsConsidered))]
A: I consider music heard by the person in terms of the following audio features: energy, mode, loudness, speechiness, instrumentalness, valence and danceability.
U: why
  [Ask(Why())]
A: I don't know.
U: ok
  [ICM(acceptance, positive)]
(A: no response)
U: so you do think this person is extraverted?
  [Ask(BooleanQuestion(Extraverted()))]
A: Yes, but I'm very uncertain.
U: sorry, I meant: why  do think this person is extraverted?
  [Ask(Why(Extraverted()))]
A: The person likes music with low energy.
U: hmm, ok
  [ICM(acceptance, positive)]
(A: no response)
U: so low energy is associated with being introverted?
  [Ask(BooleanQuestion(Supports(Not(HighValue(energy_mean)), Not(Extraverted()))))]
A: No.
U: right, my bad, low energy is associated with extraversion?
  [Ask(BooleanQuestion(Supports(Not(HighValue(energy_mean)), Extraverted())))]
A: Yes.
U: gotcha
  [ICM(acceptance, positive)]
(A: no response)
\end{dialogexample}
